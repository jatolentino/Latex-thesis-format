%%%%%%%%%%%%%%%%%%%%%%%%%%%%%%%%%%%%%%%%%%%%%%%%%%%%%%%%%%%%%%%%%%%
%  II MARCO TEÓRICO
%%%%%%%%%%%%%%%%%%%%%%%%%%%%%%%%%%%%%%%%%%%%%%%%%%%%%%%%%%%%%%%%%%%
\chapter{Marco Teórico]}

%%%%%%%%%%%%%%%%%%%%%%%%%%%%%%%%%%%%%%%%%%%%%%%%%%%%%%%%%%%%%%%%%%%
%	2.1 Estudio de ...
%%%%%%%%%%%%%%%%%%%%%%%%%%%%%%%%%%%%%%%%%%%%%%%%%%%%%%%%%%%%%%%%%%%
\section{Estudio de ...}

%%%%%%%%%%%%%%%%%%%%%%%%%%%%%%%%%%%%%%%%%%%%%%%%%%%%%%%%%%%%%%%%%%%%%%%%%%%%%%%%%%%%%%%%%%%
%	2.1.1 Ecuación de 
%%%%%%%%%%%%%%%%%%%%%%%%%%%%%%%%%%%%%%%%%%%%%%%%%%%%%%%%%%%%%%%%%%%%%%%%%%%%%%%%%%%%%%%%%%%
\subsection{Ecuación de }
\begin{adjustwidth}{2.4em}{0em}
	Para ...  de la figura \ref{fig:fig1}
	\begin{figure}[H]
		\singlespacing
		\centering
		\includegraphics[width=0.2\linewidth]{logo}
		\caption[Análisis ...]{Análisis ...}
		\label{fig:fig1}
		\flushright{
			\begin{hyphenatedparagraph}
				Fuente: Introduction to Something, Philip Peterson \& John C. Swartz , Cáp. 2, pág. 100\\
				Recuperado el 5 de junio de 2019
		\end{hyphenatedparagraph}}
	\end{figure}
	\subsubsection{Ecuación de la ...}
	Ahora se aplicará la ecuación de ... y el componente $s$ de la ecuación de ...\\
	Ecuación de partida: 
	\begin{equation*}
	\cancelto{=0(1)}{\frac{\partial}{\partial t}} \int\limits_{CV} \rho dV + \int\limits_{CS} \rho d\vec{V} \cdot d\vec{A} = 0
	\end{equation*}
	Se asume que:
	\begin{itemize}[noitemsep]
		\item El ...
		\item No existe ...
		\item El ...
	\end{itemize}
	Entonces:
	\begin{align}
	(-\rho V_s A) = -\rho(V_s+d V_s)(A+dA) \nonumber 
	\end{align}
	Expandiendo la expresión y simplificando, se tiene:
	
	\subsubsection{Componente en}
	Ecuación:

\end{adjustwidth}

%%%%%%%%%%%%%%%%%%%%%%%%%%%%%%%%%%%%%%%%%%%%%%%%%%%%%%%%%%%%%%%%%%%
%	2.2 Sensores ...
%%%%%%%%%%%%%%%%%%%%%%%%%%%%%%%%%%%%%%%%%%%%%%%%%%%%%%%%%%%%%%%%%%%
\section{Sensores ...}
\begin{adjustwidth}{2.4em}{0em}
	Esta sección ...
	\begin{itemize}[noitemsep]
		\item Los ...
		\item El ...
	\end{itemize}	
\end{adjustwidth}
